\beginsong{Rapapara}[by={Łydka Grubasa},
                     index={On był samotny, jej było źle}]
\beginverse

\[a]On był samotny, \[F]jej było źle
\[C]Gdzieś w internecie \[G]poznali się
\[a]On się zakochał \[F]ze samych zdjęć
\[C]Bo tam rusałka, \[G]dziewczę na pięć

\endverse
\beginverse

^Szczęka mu spadła ^aż pod sam stół
^Dał jej komentarz ^dziesięć i pół
^A kiedy w końcu ^spotkali się
^On jej nie poznał ^dlatego, że...

\endverse
\beginverse

^Rapapara, ^rapapara miała ^ryja jak ^kopara
^Rapapara, ^rapapara miała ^ryja jak ^kopara

\endverse
\beginverse

^On chciał zakochać ^się z całych sił
^Lecz ciągle widział ^ten wielki ryj
^W łóżku i w ^pracy, noce i dnie
Z ^hipopotamem ^kojarzył się

\endverse
\beginverse

^Rapapara, ^rapapara miała ^ryja jak ^kopara
^Rapapara, ^rapapara miała ^ryja jak ^kopara

\endverse
\beginverse

^Aż w końcu przyszedł ^zimowy czas
^Śniegu nasypało ^aż po pas
^Gdy on do pracy ^wyruszyć chciał
^Ujrzał, że w ^śniegu ugrzązł mu star

\endverse
\beginverse

^Płacząc przeklinał ^parszywy los
^Wtedy: pomogę - ^usłyszał głos
^I kiedy w starze ^zarzucał bieg
^To ona ryjem ^spychała śnieg

\endverse
\beginverse

^Rapapara, ^rapapara i tym ^ryjem jak ^kopara
^Rapapara, ^rapapara ^odkopała chłopu ^stara

\endverse
\beginverse

Ty ^przyznaj się teraz, ^przyznaj się sam
^Śmiałeś się z ryja, ^śmiałeś jak cham
I ^brałeś do ręki ^sękaty kij
I ^plułeś, i szczułeś ^ten wielki ryj

\endverse
\beginverse

Lecz ^karty rozdaje ^parszywy los
I ^ryj bywa cenny ^jak złota stos
^A więc nie śmiejcie ^się z cudzych wad
^Bo one mogą ^zbawić wasz świat

\endverse
\beginverse

^Rapapara, ^rapapara nawet ^morda jak ^kopara
^Rapapara, ^rapapara ^zasługuje na ^browara! x2

\endverse
\endsong