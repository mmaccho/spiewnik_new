
\beginsong{Я не люблю}[by={Высоцкий Владимир},
                     index={Я не люблю}]
\beginverse

Я не любл\[Am]ю фатального исх\[Dm]ода,
От ж\[G7]изни никогда не уста\[C]ю.
Я не любл\[Dm6]ю любое время года,
Когд\[F]а веселых песен не по\[E7]ю.

Я не любл\[Am]ю холодного цин\[H7]изма,
В вост\[Dm6]орженность не верю, и ещ\[E7]е -
Когда чуж\[Am]ой мои читает п\[Dm]исьма,
Загл\[E7]ядывая мне через плеч\[Am]о.

\endverse
\beginverse

Я не любл^ю, когда наполов^ину
Ил^и когда прервали разгов^ор.
Я не любл^ю, когда стреляют в спину,
Я т^акже против выстрелов в уп^ор.

Я ненав^ижу сплетни в виде в^ерсий,
Черв^ей сомненья, почестей игл^у,
Или - когд^а все время против ш^ерсти,
Ил^и - когда железом по стекл^у.

\endverse
\beginverse

Я не любл^ю уверенности с^ытой,
Уж л^учше пусть откажут тормоз^а!
Досадно мн^е, что слово "честь" забыто,
И чт^о в чести наветы за гл^аза.

Когда я в^ижу сломанные кр^ылья -
Нет ж^алости во мне и неспрост^а.
Я не любл^ю насилье и бесс^илье,
Вот т^олько жаль распятого Христ^а.

\endverse
\beginverse

Я не любл^ю себя, когда я тр^ушу,
Об^идно мне, когда невинных бь^ют,
Я не любл^ю, когда мне лезут в душу,
Тем б^олее, когда в нее плю^ют.

Я не любл^ю манежи и ар^ены,
На н^их мильон меняют по рубл^ю,
Пусть вперед^и большие перем^ены,
Я ^это никогда не полюбл^ю.

\endverse
\endsong