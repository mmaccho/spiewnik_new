\beginsong{Песня о друге}[by={Владимир Высоцкий},
                     index={Если друг оказался вдруг}]
\beginverse

Если др\[Am]уг оказался вдр\[E7]уг
И не друг, и не враг, а - \[Am]так,
Если ср\[A7]азу не разбер\[Dm]ешь,
Плох он или хор\[E7]ош, \[E]-
Парня в горы \[Dm]тяни - рискни!
Не бросай одног\[Am]о его,
Пусть он в связке в одн\[E7]ой с тоб\[E]ой -
Там пойм\[E7]ешь, кто так\[Am]ой.

\endverse
\beginverse

Если п\[Am]арень в гор\[E7]ах - не ах,
Если сразу раскис и - вн\[Am]из,
Шаг ступ\[A7]ил на ледн\[Dm]ик и - сник,
Оступился - и в кр\[E7]ик, \[E]-
Значит, рядом с тоб\[Dm]ой - чужой,
Ты его не бран\[Am]и - гони:
Вверх таких не бер\[E7]ут, и т\[E]ут
Про так\[E7]их не по\[Am]ют.

\endverse
\beginverse

Если ж ^он не скул^ил, не ныл,
Пусть он хмур был и зол, но - ш^ел,
А когд^а ты уп^ал со скал,
Он стонал, но - держ^ал, ^
Если шел за тоб^ой, как в бой,
На вершине сто^ял хмельной, -
Значит, как на себ^я самог^о,
Полож^ись на нег^о.

\endverse
\endsong