\beginsong{Песня о друге}[by={Владимир Высоцкий},
                     index={Если друг оказался вдруг}]
\beginverse

Если друг оказался вдруг
И не друг, и не враг, а - так,
Если сразу не разберешь,
Плох он или хорош, -
Парня в горы тяни - рискни!
Не бросай одного его,
Пусть он в связке в одной с тобой -
Там поймешь, кто такой.

\endverse
\beginverse

Если парень в горах - не ах,
Если сразу раскис и - вниз,
Шаг ступил на ледник и - сник,
Оступился - и в крик, -
Значит, рядом с тобой - чужой,
Ты его не брани - гони:
Вверх таких не берут, и тут
Про таких не поют.

\endverse
\beginverse

Если ж он не скулил, не ныл,
Пусть он хмур был и зол, но - шел,
А когда ты упал со скал,
Он стонал, но - держал,
Если шел за тобой, как в бой,
На вершине стоял хмельной, -
Значит, как на себя самого,
Положись на него.

\endverse
\endsong