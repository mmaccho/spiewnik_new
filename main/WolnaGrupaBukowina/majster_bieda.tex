\beginsong{Majster bieda}[by={Wolna Grupa Bukowina},
                     index={Skąd przychodził, kto go znał}]
\beginverse

Skąd przychodził, kto go znał
Kto mu rękę podał kiedy
Nad rowem siadał, wyjmował chleb
Serem przekładał i dzielił się z psem
Tyle wszystkiego, co sobą miał
Majster Bieda

\endverse
\beginverse

Czapkę z głowy ściągał, gdy
Wiatr gałęzie chylił drzewom
Śmiał się do słońca i śpiewał do gwiazd
Drogą bez końca co przed nim szła
Znał jak pięć palców, jak szeląg zły
Majster Bieda

\endverse
\beginverse

Nikt nie pytał skąd się wziął
Gdy do ognia się przysiadał
Wtulał się w krąg ciepła jak w kożuch
Zmęczony drogą wędrowiec boży
Zasypiał długo gapiąc się w noc
Majster Bieda

\endverse
\beginverse

Aż nastąpił taki rok
Smutny rok, tak widać trzeba
Nie przyszedł Bieda zieloną wiosną
Miejsce, gdzie siadał, zielskiem zarosło
I choć niejeden wytężał wzrok
Choć lato pustym gościńcem przeszło
Z rudymi liśćmi jesienną schedą
Wiatrem niesiony popłynął w przeszłość /x3
Majster Bieda

\endverse
\endsong