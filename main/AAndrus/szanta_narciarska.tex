\beginsong{Szanta narciarska}[by={Artur Andrus},
                     index={Nazywali go marynarz}]
\beginverse

Nazywali go marynarz,
Bo opaskę miał na oku.
Na każdym stoku dziewczyna,
Dziewczyna na każdym stoku.
Pochodzi spod Poznania,
Podobno umie wróżyć z kart.
Panny rwie na wiązania,
Mężatki - na długość nart.

\endverse
\beginverse

Caryco mokrego śniegu
Ratrakiem płynę do Ciebie pod prąd.
Hej!
Dobrze, że stoisz na brzegu,
Bo ja właśnie schodzę na ląd.

\endverse
\beginverse

Nigdy się nie lękał biedy
I się nie przejmował jutrem.
A jego ratrak był kiedyś
Zwyczajnym rybackim kutrem.
I woził dorsze i rybole.
Zimą i latem, okrągły rok.
Teraz jak nieraz przejedzie
Rybolami czuć cały stok.

\endverse
\beginverse

Caryco mokrego śniegu...

\endverse
\beginverse

Wszyscy w porcie odetchnęli.
Zwiał nim się zakończył sezon.
Jeszcze się tam jak żagiel bieli
Jego czarny kombinezon.
Odpłynął pod Ustrzyki
I przez kobiety wpadł w kłopoty.
Forsę z polowań na orczyki
Przehulał na antybiotyk.

\endverse
\beginverse

Caryco mokrego śniegu...

\endverse
\beginverse

Jeśli kiedyś go zobaczysz
Na ratraku w podłym świecie,
To powiedz mu, że w Karpaczu
Czekają na niego dzieci.
I kiedy opuszcza statek,
Żeby się znowu oddać złu,
Każda z dwudziestu siedmiu matek
Dzieciątku śpiewa do snu:

\endverse
\beginverse

Caryco mokrego śniegu... x2

\endverse
\endsong