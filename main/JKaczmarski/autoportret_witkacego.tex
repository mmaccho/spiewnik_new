\beginsong{Autoportret Witkacego}[by={Jacek Kaczmarski},
                     index={Patrzę na świat z nawyku}]
\beginverse

Patrzę na świat z nawyku
Więc to nie od narkotyków
Mam czerwone oczy doświadczalnych królików
Wstałem właśnie od stołu
Więc to nie z mozołu
Mam zaciśnięte wargi zgłodniałych Mongołów

\endverse
\beginverse

Słucham nie słów lecz dźwięków
Więc nie z myśli fermentu
Mam odstające uszy naiwnych konfidentów
Wszędzie węszę bandytów
Więc nie dla kolorytu
Mam typowy cień nosa skrzywdzonych Semitów

\endverse
\beginverse

Widzę kształt rzeczy w ich sensie istotnym
I to mnie czyni wielkim oraz jednokrotnym
W odróżnieniu od was którzy Państwo wybaczą
Jesteście wierszem idioty odbitym na powielaczu

\endverse
\beginverse

Dosyć sztywną mam szyję
I dlatego wciąż żyję
Że polityka dla mnie to w krysztale pomyje
Umysł mam twardy jak łokcie
Więc mnie za to nie kopcie
Że rewolucja dla mnie to czerwone paznokcie

\endverse
\beginverse

Wrażliwym jest jak membrana
Zatem w wieczór i z rana
Trzęsę się jak śledziona z węgorza wyrwana
Zagłady świata się boję
Więc dla poprawy nastroju
Wrzeszczę jak dziecko w ciemnym zamknięte pokoju

\endverse
\beginverse

Ja bardziej niż wy jeszcze krztuszę się i duszę
Ja częściej niż wy jeszcze żyć nie chcę a muszę
Ale tknąć się nikomu nie dam i dlatego
Gdy trzeba będzie sam odbiorę światu Witkacego

\endverse
\endsong