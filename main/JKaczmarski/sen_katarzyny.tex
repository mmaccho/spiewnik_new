\beginsong{Sen Katarzyny II}[by={Jacek Kaczmarski},
                     index={Na smyczy trzymam filozofów Europy}]
\beginverse

Na smyczy \[G]trzymam \[D]filozofów \[G]Europy
Podparłam \[G]armią \[D]marmurowe Piotra \[e]stropy
Mam psy, \[C]sokoły, konie, \[D]kocham łów \[e]szalenie
A wokół \[C]same \[D]zające i \[G]jelenie
\[Fis]Pałace stawiam głowy \[h]ścinam
\[Fis]Kiedy mi przyjdzie na to \[G D]chęć
\[C]Mam \[D]biografów, \[e]portrecistów
I jeszcze \[C]jedno \[D]pragnę \[G]mieć...

\endverse
\beginverse

\[e]Stój \[h]Katarzyno! \[e]koronę \[h]carów
\[e]Sen taki jak \[a]ten może ci z \[C D]głowy \[G]zdjąć

\endverse
\beginverse

Kobietą \[G]jestem ponad \[D]miarę swoich \[G]czasów
Nie bawią \[G]mnie umizgi \[D]bladych \[e]lowelasów
Ich miękkich \[C]palców dotyk \[D]budzi \[e]obrzydzenie
Już wolę \[C]łowić \[D]zające i \[G]jelenie
\[Fis]Ze wstydu potem ten i \[h]ów
\[Fis]Rzekł o mnie: niewyżyta \[G D]Niemra
\[C]I pod \[D]batogiem nago \[e]biegł
Po \[C]śniegu \[D]dookoła \[G]Kremla

\endverse
\beginverse

Stój Katarzyno...

\endverse
\beginverse

Kochanka ^trzeba mi ^takiego jak ^imperium
Co by mnie ^brał tak, jak ja ^daję: całą ^pełnią
Co by i ^władcy i ^poddańca był ^wcieleniem
By mi ^zastąpił ^zające i ^jelenie
^Co by rozumiał tak jak ^ja
^Ten głupi dwór rozdanych ^ról
^I pośród ^pochylonych ^głów
^Dawał mi ^rozkosz albo ^ból

\endverse
\beginverse

\[e]Stój \[h]Katarzyno! \[e]koronę \[h]carów
\[e]Sen taki jak \[a]ten może ci z \[C D]głowy \[G]zdjąć
\[e]Gdyby się \[h]kiedyś \[e]kochanek taki \[h]znalazł...
\[e]Wiem, sama \[a]wiem! \[C]Kazałabym \[D]go \[G]ściąć!

\endverse
\endsong