\beginsong{Nasza klasa}[by={Jacek Kaczmarski},
                     index={Co się stało z naszą klasą}]
\beginverse

Co się stało z naszą klasą
Pyta Adam w Tel-Avivie,
Ciężko sprostać takim czasom,
Ciężko w ogóle żyć uczciwie -
Co się stało z naszą klasą?
Wojtek w Szwecji, w porno klubie
Pisze - dobrze mi tu płacą
Za to, co i tak wszak lubię.

\endverse
\beginverse

Kaśka z Piotrkiem są w Kanadzie,
Bo tam mają perspektywy,
Staszek w Stanach sobie radzi,
Paweł do Paryża przywykł,
Gośka z Przemkiem ledwie przędą,
W maju będzie trzeci bachor,
Próżno skarżą się urzędom,
Że też chcieli by na zachód,

\endverse
\beginverse

Za to Magda jest w Madrycie
I wychodzi za Hiszpana,
Maciek w grudniu stracił życie,
Gdy chodzili po mieszkaniach,
Janusz, ten, co zawiść budził,
Że go każda fala niesie,
Jest chirurgiem, leczy ludzi,
Ale brat mu się powiesił,

\endverse
\beginverse

Marek siedzi za odmowę,
Bo nie strzelał do Michała,
A ja piszę ich historię
I to już jest klasa cała.
Jeszcze Filip, fizyk w Moskwie -
Dziś nagrody różne zbiera,
Jeździ, kiedy chce do Polski,
Był przyjęty przez premiera.

\endverse
\beginverse

Odnalazłem klasę całą -
Na wygnaniu, w kraju, w grobie,
Ale coś się pozmieniało,
Każdy sobie żywot skrobie -
Odnalazłem całą klasę
Wyrośniętą i dojrzałą,
Rozdrapałem młodość naszą,
Lecz za bardzo nie bolało...

\endverse
\beginverse

Już nie chłopcy, lecz mężczyźni,
Już kobiety - nie dziewczyny.
Młodość szybko się zabliźni,
Nie ma w tym niczyjej winy;
Wszyscy są odpowiedzialni,
Wszyscy mają w życiu cele,
Wszyscy w miarę są - normalni,
Ale przecież - to niewiele...

\endverse
\beginverse

Nie wiem sam, co mi się marzy,
Jaka z gwiazd nade mną świeci,
Gdy wśród tych - nieobcych - twarzy
Szukam ciągle twarzy - dzieci,
Czemu wciąż przez ramię zerkam,
Choć nie woła nikt - kolego!
Że ktoś ze mną zagra w berka,
Lub przynajmniej w chowanego...

\endverse
\beginverse

Własne pędy, własne liście,
Zapuszczamy - każdy sobie
I korzenie oczywiście
Na wygnaniu, w kraju, w grobie,
W dół, na boki, wzwyż ku słońcu,
Na stracenie, w prawo - w lewo...
Kto pamięta, że to w końcu
Jedno i - to samo drzewo...

\endverse
\endsong