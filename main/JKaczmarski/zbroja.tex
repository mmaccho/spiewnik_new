\beginsong{Zbroja}[by={Jacek Kaczmarski},
                     index={Dałeś mi Panie zbroję}]
\beginverse

Dałeś mi Panie zbroję, dawny kuł płatnerz ją
W wielu pogięta bojach, w wielu ochrzczona krwią
W wykutej dla giganta potykam się co krok
Bo jak sumienia szantaż uciska lewy bok

\endverse
\beginverse

Lecz choć zaginął hełm i miecz
Dla ciała żadna w niej ostoja
To przecież w końcu ważna rzecz
Zbroja

\endverse
\beginverse

Magicznych na niej rytów dziś nie odczyta nikt
Ale wykuta z mitów i wieczna jest jak mit
Do ciała mi przywarła, nie daje żyć i spać
A tłum się cieszy z karła, co chce giganta grać

\endverse
\beginverse

Lecz choć zaginął...

\endverse
\beginverse

A taka w niej powaga dawno zaschniętej krwi
Że czuję jak wymaga i każe rosnąć mi
Być może nadaremnie, lecz stanę w niej za stu
Zdejmij ją Panie ze mnie, jeśli umrę podczas snu

\endverse
\beginverse

Bo choć zaginął...

\endverse
\beginverse

Wrzasnęli hasło "wojna", zbudzili hufce hord
Zgwałcona noc spokojna ogląda pierwszy mord
Goreją świeże rany, hańbiona płonie twarz
Lecz nam do obrony dany pamięci pancerz nasz

\endverse
\beginverse

Choć, choć za ciosem pada cios
I wróg posiłki śle w konwojach
Nas przed upadkiem chroni wciąż
Zbroja

\endverse
\beginverse

Wywlekli pudła z blachy, natkali kul do luf
I straszą sami w strachu, strzelają do ciał i słów
Zabrońcie żyć wystrzałem, niech zatryumfuje gwałt
Nad każdym wzejdzie ciałem pamięci żywej kształt

\endverse
\beginverse

Choć słońce skrył bojowy gaz
I żołdak pławi się w rozbojach
Wciąż przed upadkiem chroni nas
Zbroja

\endverse
\beginverse

Wytresowali świnie, kupili sobie psy
I w pustych słów świątyni stawiają ołtarz krwi
Zawodzi przed bałwanem półślepy kapłan-łgarz
I każdym nowym zdaniem hartuje pancerz nasz

\endverse
\beginverse

Choć krwią zachłysnął się nasz czas
Choć myśli toną w paranojach
Jak zawsze chronić będzie nas
Zbroja

\endverse
\endsong